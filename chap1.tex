\chapter{The First Proof}
\chaps{The First Proof}

In this chapter we will recount the way in which Menger's Theorem was presented in the Part B lecture course and the proof as an application of the max-flow-min-cut theorem.

\section{Measures of Connectedness}

Menger's Theorem is an equality between two measures of connectedness within a graph, in this section we build to the definitions of the measures required to state the theorem.

\begin{defn}
For a graph $G$, a set $S \subseteq V(G)$ \textbf{separates} $G$ if the graph $G-S$ obtained by removing the vertices of $S$ and all incident edges from $G$ is disconnected.
\end{defn}

For example, when $G$ is the cyclic graph $C_n$ any non-empty set $S \subseteq V(G)$ separates $G$, and in the case where $G = K_3$, then it is clear that any set $S$ with $|S| \geq 2$ will be separating.

\begin{defn}
A graph $G$ with $|G| > k$ is $\boldsymbol{k}$\textbf{-connected} if there is no set $S \subseteq V(G)$ with $|S| = k - 1$ such that $S$ separates $G$.
\end{defn}

Observe that this definition entails that every graph is $0$-connected and that a graph is $1$-connected if and only if it is a connected graph with at least two vertices. Thus, $k$-connectedness gives a generalization of the concept of a connected graph, and the maximum $k$ such that a graph is $k$-connected - or, equivalently, the size of a minimal separating set - provides a natural measure of the connectedness of a graph. 

\begin{notat}
We denote by $\kappa(G)$ the maximum $k$ such that $G$ is $k$-connected
\end{notat}

$\kappa(G)$ gives us a measure of the connectedness of a graph on a global scale. We now extend this to a local measure of how connected two given points in the graph are with relation to removing vertices. Since two adjacent vertices - vertices joined by a single edge - can never be disconnected by removing other vertices we restrict ourselves to non-adjacent vertices and the following definition is as expected.

\begin{defn}
A set $S \subseteq V(G)$ separates distinct, non-adjacent vertices $x$ and $y$ in $G$ if $x$ and $y$ are in different connected components of $G - S$
\end{defn}

\begin{notat}
We denote by $\kappa_G(x,y)$, or simply $\kappa(x,y)$, the minimum size of a set $S$ separating $x$ and $y$
\end{notat} 

The statement that $x$ and $y$ are in different connected components of $G-S$ is equivalent to saying that there are no paths between $x$ and $y$ in the residual graph $G-S$. In lay terms one might measure the connectedness of two cities by how many different ways there are of travelling between them. For example in Figure \ref{fig:towns}

\begin{figure}[h]
\label{fig:towns}
\begin{center}
\psscalebox{1 1} % Change this value to rescale the drawing.
{
\begin{pspicture}(0,-2.0)(8.394231,2.0)
\psdots[linecolor=black, dotsize=0.5](0.19711548,0.0)
\psdots[linecolor=black, dotsize=0.5](4.1971154,0.0)
\psdots[linecolor=black, dotsize=0.5](8.197116,0.0)
\psellipse[linecolor=black, linewidth=0.04, dimen=outer](2.1971154,0.0)(2.0,0.4)
\psellipse[linecolor=black, linewidth=0.04, dimen=outer](2.1971154,0.0)(2.0,0.8)
\psellipse[linecolor=black, linewidth=0.04, dimen=outer](2.1971154,0.0)(2.0,1.2)
\psellipse[linecolor=black, linewidth=0.04, dimen=outer](2.1971154,0.0)(2.0,1.6)
\psellipse[linecolor=black, linewidth=0.04, dimen=outer](2.1971154,0.0)(2.0,2.0)
\psellipse[linecolor=black, linewidth=0.04, dimen=outer](6.1971154,0.0)(2.0,2.0)
\psellipse[linecolor=black, linewidth=0.04, dimen=outer](6.1971154,0.0)(2.0,1.6)
\psellipse[linecolor=black, linewidth=0.04, dimen=outer](6.1971154,0.0)(2.0,0.8)
\psellipse[linecolor=black, linewidth=0.04, dimen=outer](6.1971154,0.0)(2.0,1.2)
\psellipse[linecolor=black, linewidth=0.04, dimen=outer](6.1971154,0.0)(2.0,0.4)
\psline[linecolor=black, linewidth=0.04](0.17711549,0.02)(8.177115,0.02)(8.197116,-0.02)
\rput[bl](0.02,-0.12){\textcolor{white}{\small\textbf{A}}}
\rput[bl](4.01,-0.12){\textcolor{white}{\small\textbf{B}}}
\rput[bl](7.99,-0.12){\textcolor{white}{\small\textbf{C}}}
\end{pspicture}
}
\end{center}

\caption{Demonstrating levels of connectedness}
\end{figure}


\begin{defn}
A pair of $x-y$ paths are \textbf{independent} if the only vertices they share are $x$ and $y$. A set of $x-y$ paths is independent if the paths are pairwise independent.
\end{defn}

As before the idea of connectedness we are seeking to capture is tied in with a dual notion of separation, for if we were to remove the interior (all vertices except the ends $x$ and $y$) of each of the paths in a maximally independent set of $x-y$ paths then we would disconnect $x$ and $y$ within the graph - there would be no path between them in the remainder, else we could add it to our set of paths contradicting the sets maximality. Since adjacent vertices - those joined by a single edge in the graph - can never be disconnected by removing other vertices from the graph we exclude such vertices from our definition. 

\begin{notat}
hi
\end{notat}



\newpage



\thm[Menger's Theorem]{
Let $x$ and $y$ be distinct non-adjacent vertices of
a graph $G$. Then the maximum size of an independent set of $x–y$ paths is $\kappa_G(x, y)$.
}